% abstract_eng.tex
% グループ報告書全体の概要について(English)
% ---Source---
\thispagestyle{plain}
\begin{comment}
プロジェクトで解決した問題点、設定した課題、
および解決策の概要を簡潔に記述。
\end{comment}
%
\section*{
\begin{center}
Abstract
\end{center}}
\noindent
So far, clerk robots have been introduced into actual stores in various ways.
Recently, there is an example of a trial introduction of a clerk robot that can display products remotely at FamilyMart.
There are various motives for introducing a clerk robot, such as reducing labor shortages, avoiding contact between people, and expanding the human body to operate remotely.
In this way, while the clerk robot has many usefulness, its introduction to the field is still a high hurdle.
From the perspective of robot-human interaction, in order to eliminate the "inorganic nature of robots," which is thought to be one of the causes.
By redesigning better communication between robots and humans and working to realize it, we have developed a robot that can be easily introduced into the field.
Also, in developing a useful clerk robot,
By analyzing and clarifying the role of the interaction between the clerk and the customer in the actual store,
Clarified the role that the clerk robot must fulfill.
In that role, we aimed to realize "improvement of customers' purchasing motivation" and "improvement of comfort in the store", which are particularly related to interaction.
In order to improve customers' purchasing motivation, we introduced a function to introduce recommended products.
In product introduction as well, in order to eliminate the inorganic nature of the robot, we focused on product introduction in which the customer peeks at the thought that the robot wants the product.
In order to improve the comfort of the store, we introduced a function to say hello.
In order to eliminate the inorganic nature of the robot in greetings, the robot senses the customer's entry and greets himself accordingly.
We reduced the waiting movements that are typical of robots that always wait for customer input, and focused on spontaneous movements that are human-like.
In addition, in order to promote communication between customers and robots, we have prepared a "stroking function" that realizes interaction through physical contact.
Finally, we worked to realize "asynchronous operation" as a function that sets it apart from existing robot-type interfaces.
Asynchronous operation is a function that allows a robot to kill time in a human-like manner in a waiting state where there is no customer input.
The appearance that the robot is stuck waiting for input promotes the inorganic impression of the robot, and I thought that it was important to eliminate it.
In order to realize these services, the limitation due to hardware performance becomes a serious problem in the method of extending the existing robot-type interface.
Therefore, in this project this year, we adopted a method of developing from a robot-type interface in order to flexibly develop a clerk robot from both hardware and software perspectives.
\\\\
\noindent
{\bf\gt Keyword} Arduino, robot-type interface, communication
\begin{flushright}
(※文責:伊藤壱)
\end{flushright}