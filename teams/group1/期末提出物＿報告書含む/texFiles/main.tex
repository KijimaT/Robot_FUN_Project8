% ドキュメントクラス[オプション]{文書クラス}
% 文書クラス:和文(新)
% オプション:dvipdfmx(pdf出力をオンにする)
% オプション:autodetect-engine(下に説明)
% オプション:titlepage(タイトルが表紙として扱われる)
%エンジン自動オプション。
%文書から使用するtexエンジンを自動で決定する)
\documentclass[dvipdfmx, autodetect-engine, titlepage, 10pt, a4paper]{jsreport}
% 図形描画パッケージ
\usepackage{tikz}
% コメントアウトのパッケージ
\usepackage{comment}
% ヘッダー・フッターを管理するパッケージ
\usepackage{fancyhdr}
% ヘッダー・フッターのページスタイル
\pagestyle{fancy}
\lhead{Practical application of the robot-type interaction} %ヘッダ左
\chead{} %ヘッダ中央
\rhead{} %ヘッダ右.コンパイルした日付を表示
\lfoot{Group Report of 2020 SISP} %フッタ左(このSISPとは)
\cfoot{\thepage} %フッタ中央.ページ番号を表示
\rfoot{Group Number 8} %フッタ右
\renewcommand{\headrulewidth}{0pt} %ヘッダの罫線
% 余白の設定
%\usepackage[margin=3cm]{geometry}
% 見出しページでもヘッダ・フッタの設定を適用させる
\makeatletter
\renewcommand{\chapter}{%
    \if@openleft\cleardoublepage\else
    \if@openright\cleardoublepage\else\clearpage\fi\fi
    %\plainifnotempty %元: \thispagestyle{plain}
    \global\@topnum\z@
    \if@english \@afterindentfalse \else \@afterindenttrue \fi
    \secdef
    {\@omit@numberfalse\@chapter}%
    {\@omit@numbertrue\@schapter}}
\makeatother
% sectionに行間を詰める引数を設定できるようにする
\newcommand{\beforesection}[1]{
 \vspace{-#1mm}
}
% 1行に含む文字の数と、1ページに含む行数の指定をするための拡張
\makeatletter
\def\mojiparline#1{
\newcounter{mpl}
\setcounter{mpl}{#1}
\@tempdima=\linewidth
\advance\@tempdima by-\value{mpl}zw
\addtocounter{mpl}{-1} \divide\@tempdima by \value{mpl}
\advance\kanjiskip by\@tempdima
\advance\parindent by\@tempdima
}
\makeatother
\def\linesparpage#1{
\baselineskip=\textheight
\divide\baselineskip by #1
} 
%
%
%
\begin{document}
\thispagestyle{empty}
\linesparpage{42} % 1ページあたり行数の指定
\mojiparline{43} % 一行あたり文字数の指定
% 本文まではアラビア数字でページ番号を割り当てなおす
\pagenumbering{roman}
% ---タイトル---
\thispagestyle{plain}
\begin{titlepage}
\begin{center}
{\LARGE\bf 公立はこだて未来大学2020年度システム情報科学実習
\\ グループ報告書}
\vspace{5mm}\\
{\Large\bf Future University Hakodate 2020 System Information Science Project Group Report}
\vspace{5mm} \\
{\Large\bf プロジェクト名}\\
{\Large ロボット型ユーザインタラクションの実用化\\-「未来大発の店員ロボット」をハードウエアから開発する-}
\vspace{2mm}\\
{\Large\bf Project Name}\\
{\Large Practical application of the robot-type interaction}
\vspace{5mm} \\
{\large\bf プロジェクト番号/Project No.}\\
{\large 8}
\vspace{3mm} \\
{\large\bf グループ名}\\
{\large グループA}
\vspace{2mm} \\
{\large\bf Group Name}\\
{\large Group A}
\vspace{5mm} \\
{\large\bf プロジェクトリーダ/Project Leader}\\
\begin{tabular}{lll}
1018194 & 伊藤壱 & Hajime Ito\\
\end{tabular}
\vspace{5mm} \\
{\large\bf グループリーダ/Group Leader}\\
\begin{tabular}{lll}
1018167 & 宮嶋佑 & Tasuku Miyajima\\
\end{tabular}
\vspace{5mm} \\
{\large\bf グループメンバ/Group Member}\\
\begin{tabular}{lll}
1018194 & 伊藤壱 & Hajime Ito\\
1018239 & 木島拓海 & Takumi Kijima\\
1018103 & 藤内悠 & Haruka Fujiuchi\\
1018167 & 宮嶋佑 & Tasuku Miyajima\\
\end{tabular}
\vspace{5mm} \\
{\large\bf 指導教員/Advisor}\\
\begin{tabular}{ll}
三上貞芳 & Sadayoshi Mikami \\
鈴木昭二 & Sho'ji Suzuki\\
高橋信行 & Nobuyuki Takahashi\\
\end{tabular}
\vspace{5mm} \\
{\large\bf 提出日}\\
{\large 2020年9月23日}\vspace{2mm} \\
{\large\bf Date of Submission}\\
{\large September 23, 2020}
\end{center}
\end{titlepage}
% 字下げ込みでの指定になるので、字下げ分1少なく設定する
% 概要より本文優先の字数設定。
% ---概要---
\thispagestyle{plain}
% abstract.tex
% グループ報告書全体の概要について
% ---Source---
\thispagestyle{plain}
\begin{comment}
プロジェクトで解決した問題点、設定した課題、
および解決策の概要を簡潔に記述。
\end{comment}
%
\section*{
\begin{center}
概要
\end{center}}
\noindent
これまでに様々な形で実店舗において店員ロボットが導入されてきた.
最近ではファミリーマートで遠隔操作で商品の陳列ができる店員ロボットの試験導入が行われたという例がある.
店員ロボットを導入する動機は様々であるが,人手不足の軽減や、人と人の接触を避けること、遠隔で操作する人間の身体拡張などがある.
このように店員ロボットは数多くの有用性を持つ一方で,現場への導入は未だにハードルの高いものとなっているのが現状だ.
その原因の一つであると考えられているのが「ロボットの無機質さ」である.ロボットに不慣れな人はロボットに対して,怖い,もしくは不気味という印象を抱いてしまう.そのような問題を解消すべく,ロボットと人間のインタラクション(相互行為)という観点から
ロボットと人間のより良いコミュニケーションを再設計し,その実現に取り組むことで現場へ導入し易いロボットの開発を行った.
また,有用な店員ロボットを開発するにあたって,
実店舗における店員と顧客とのインタラクションが持つ役割を分析し明確にすることで,
店員ロボットが実現しなければならない役割を明らかにした.
その役割において,特にインタラクションに関連する部分である「顧客の購買意欲の向上」や,「店舗内の居心地の良さの向上」の実現を目指した.
顧客の購買意欲向上の実現にあたって,お勧め商品の紹介を行う機能を導入した.
商品紹介においてもロボットの無機質を解消するため,ロボットが該当商品を欲しいと考えている思考を顧客が覗き見る形での商品紹介にこだわった.
店舗内の居心地の良さ向上のためには,挨拶を行う機能を導入した.
挨拶においてもロボットの無機質さを解消するため,ロボットが顧客の入店を感知し,それに合わせて自分から挨拶を行うことで,
常に顧客の入力を待つロボットらしい待ち動作を減らして人間らしい自発的な動作にこだわった.
加えて,顧客とロボットのコミュニケーション促進を図るため,身体的接触によるインタラクションを実現する「撫でられ機能」を用意した.
最後に,既存のロボット型インタフェースとは一線を画す機能として「非同期動作」の実現に努めた.
非同期動作とは顧客の入力がない待ち状態において,ロボットに人間らしい暇つぶしを行わせる機能である.
ロボットが入力待ちで固まっている様子はロボットの無機質な印象を助長するものであり,その解消が重要であると考えた.
これらのサービスを実現するには,既存のロボット型インタフェースを拡張する手法ではハードウェア性能による制約が重大な問題点になってしまう.
従って今年度における本プロジェクトではハードウェアとソフトウェアの両面から柔軟に店員ロボットを開発すべく,ロボット型インタフェースから開発する手法をとった.
\\\\
\noindent
{\bf\gt キーワード} Arduino, ロボット型インタフェース, コミュニケーション
\begin{flushright}
(※文責:伊藤壱)
\end{flushright}
\thispagestyle{plain}
% abstract_eng.tex
% グループ報告書全体の概要について(English)
% ---Source---
\thispagestyle{plain}
\begin{comment}
プロジェクトで解決した問題点、設定した課題、
および解決策の概要を簡潔に記述。
\end{comment}
%
\section*{
\begin{center}
Abstract
\end{center}}
\noindent
So far, clerk robots have been introduced into actual stores in various ways.
Recently, there is an example of a trial introduction of a clerk robot that can display products remotely at FamilyMart.
There are various motives for introducing a clerk robot, such as reducing labor shortages, avoiding contact between people, and expanding the human body to operate remotely.
In this way, while the clerk robot has many usefulness, its introduction to the field is still a high hurdle.
From the perspective of robot-human interaction, in order to eliminate the "inorganic nature of robots," which is thought to be one of the causes.
By redesigning better communication between robots and humans and working to realize it, we have developed a robot that can be easily introduced into the field.
Also, in developing a useful clerk robot,
By analyzing and clarifying the role of the interaction between the clerk and the customer in the actual store,
Clarified the role that the clerk robot must fulfill.
In that role, we aimed to realize "improvement of customers' purchasing motivation" and "improvement of comfort in the store", which are particularly related to interaction.
In order to improve customers' purchasing motivation, we introduced a function to introduce recommended products.
In product introduction as well, in order to eliminate the inorganic nature of the robot, we focused on product introduction in which the customer peeks at the thought that the robot wants the product.
In order to improve the comfort of the store, we introduced a function to say hello.
In order to eliminate the inorganic nature of the robot in greetings, the robot senses the customer's entry and greets himself accordingly.
We reduced the waiting movements that are typical of robots that always wait for customer input, and focused on spontaneous movements that are human-like.
In addition, in order to promote communication between customers and robots, we have prepared a "stroking function" that realizes interaction through physical contact.
Finally, we worked to realize "asynchronous operation" as a function that sets it apart from existing robot-type interfaces.
Asynchronous operation is a function that allows a robot to kill time in a human-like manner in a waiting state where there is no customer input.
The appearance that the robot is stuck waiting for input promotes the inorganic impression of the robot, and I thought that it was important to eliminate it.
In order to realize these services, the limitation due to hardware performance becomes a serious problem in the method of extending the existing robot-type interface.
Therefore, in this project this year, we adopted a method of developing from a robot-type interface in order to flexibly develop a clerk robot from both hardware and software perspectives.
\\\\
\noindent
{\bf\gt Keyword} Arduino, robot-type interface, communication
\begin{flushright}
(※文責:伊藤壱)
\end{flushright}
% ---目次の表示---
\thispagestyle{plain}
\tableofcontents 
% ---本文---
% ---第1章---
% はじめに
% introduction.tex
% 第1章「はじめに」の内容
% ---Source---
\chapter{はじめに} 
% 始めにとまとめを読んだだけで、全体が把握できるように記述する
%
この章では,現在のロボット型インターフェイスの現状とその問題,本プロジェクトで作成するロボット型インターフェイスの目的について記述する.
\begin{flushright}
(※文責:伊藤壱)
\end{flushright}
% 本文からはアラビア数字でページ番号を割り当てなおす。
\pagenumbering{arabic}
\section{ロボット型インタフェース}
\subsection{現状}
%
%
新型コロナウィルスの蔓延による社会情勢において,オンライン会議に代表されるような非接触のコミュニケーションが推進されている.小売業の実店舗において,人間同士の接客サービスが避けられる中,その代替手段としてロボットの接客に注目が集まっている.
これは,小売業の人手不足による店員ロボットの導入に拍車をかける形で需要を増やしている.
このような状況を受けて,ロボット産業の市場規模は2035年までに5倍となる見通しも出ている.
産業ロボットといえば,工場のオートメーション化に用いられるロボットが想起されるが,実際のところ、現在では店員ロボット導入の試みが至る所で行われている.
しかし,店員ロボットの分野はまだまだ未発達であり,ロボットの性能の問題や,顧客がロボットに馴染めないという意識的な問題を抱えている.
その理由として,店員ロボットは人間が働く環境で人間の行う仕事をそのまま引き受けて働くという状況にあり,工場で稼働されるような他の産業用ロボットよりも汎用的な性能を求められる点がある.
そのような店員ロボットの性能の問題を,人間が遠隔で操作をするという形で乗り越えるのが近年のトレンドになっているアバターロボットである.
アバターロボットとは,人間の身体の拡張であると捉えることが出来る.
しかし,現状でのアバターロボットは人間の指示の通りに動作をするのみであり,無機質さの解消には至っていない.
もし,人間に親しまれる機能をもつロボット型インタフェースを開発することが出来れば店員ロボットの導入の増加が期待されるほか,トレンドとなっているアバターロボットの実用性をさらに高めることが出来るだろう.
そこで私たちは,ロボットの持つ無機質さの解消に努め,顧客が馴染みやすい店員ロボットの開発を目指すことにした.
\begin{flushright}
(※文責:伊藤壱)
\end{flushright}
% プロジェクトの背景を記述する
\subsection{現状の問題}
%
%
産業界で工場などに導入されるロボットは生産技術を担う技能職の代替としての仕事が期待されるが,小売り業の実店舗などに導入される店員ロボットは接客を行うサービス職の代替としての仕事が期待される.
一般にロボットは無機質な外観をしており,人と同じかそれ以上大きく,素材も金属であったりする.
そのため,ロボットに対して威圧感を感じる人も少なくない.
これは,サービス業を担う店員ロボットとしては重大な問題である.
また,店員ロボット用に開発された機体においても外観の問題は解決されているが,人間からの入力がない場合に固まってしまったり,同じパターンの動作を無機質に繰り返してしまうという問題が見受けられる.
\begin{flushright}
(※文責:伊藤壱)
\end{flushright}
%\subsection{従来例}
\begin{comment}
現在の該当分野や類似プロジェクトの状況を記述する.
昨年のテーマを引き継いでいるプロジェクトでは、昨年の内容も記述する.
\end{comment}
%
%

%\begin{flushright}
%(※文責:伊藤壱)
%\end{flushright}
%\subsection{従来の問題}
%
%
\begin{comment}
目的を妨げている問題(現在の該当分野や類似プロジェクトの問題点)を記述する.
\end{comment}
%\begin{flushright}
%(※文責:伊藤壱)
%\end{flushright}
%\subsection{課題}
\begin{comment}
1.4節の従来の問題点の中から,解決すべき問題を明示し,その問題を解決するために設定された課題(解決すべき問題を具体的に記述したもの)の概略を記述する.
\end{comment}
%
%
%\begin{flushright}
%(※文責:伊藤壱)
%\end{flushright}
\section{今回開発したロボット型インタフェース}
まだ開発していないため,記述できません.
\begin{flushright}
(※文責:伊藤壱)
\end{flushright}
\section{目的}
\begin{comment}
プロジェクトの目的を記述する.ここでの目的とは,最も高位のものであり,意思や希望を表す.
プロジェクトのテーマの説明にもなるように記述する.
\end{comment}
%
%
1.1.1項で述べた通り,小売業などでの店員ロボットの導入は進んでおり,アバターロボットのような形態での導入がトレンドになりつつある.しかし、1.1.2項で述べたようにロボットの持つ無機質さの解消が大きい課題となっている.そこで本プロジェクトでは,有用性のある店員ロボットを目指しつつ,人間が馴染みやすいロボットの動き,機能の充実を実現する.
\begin{flushright}
(※文責:伊藤壱)
\end{flushright}
% ---第2章---
% プロジェクトの概要
% project_abs.tex
% 第2章「プロジェクトの概要」の内容
% ---Source---
\chapter{プロジェクトの概要}
この章では,第1章を元に問題と課題の設定,それらの到達目標,到達目標を達成するための各個人の割り当てについて述べている.
\section{問題の設定}
1.1.2節で述べた既存のロボットの問題を以下のようにまとめた.
\begin{itemize}
    \item 外観が無機質で,威圧的に感じてしまう場合がある
    \item 客からの入力がない場合,静止する状態が続く.または,同じパターンの動きを繰り返してしまい,不気味に見えてしまう.
\end{itemize}
我々のグループは2つ目のロボットの「動き」に関わる問題に重点を置き活動をする.
\begin{flushright}
(※文責:宮嶋佑)
\end{flushright}
\section{課題の設定}
2.1節で述べた問題を,以下の制約条件下で解決することを考えた。
 \begin{itemize}
\item コロナウイルス感染対策を念頭に置き,対面での活動は必要最低限にする.
\item 低予算でかつ,高効率で安全に活動できる
\item 無意味な作業をなくす.
\item 大学の講義内で得た知識,技術を生かす.
\item 新たに学習を行い,大学の講義では得ることのできない知識,技術の習得も行う.
\end{itemize}
その結果,以下の具体策が提案された.
 \begin{itemize}
\item 各個人の作業を分担することで,作業の明確化を行う
\item   備品を購入する際には,金額や必要性をよく考え,自身の判断だけでなく,グループメンバーにも確認を取ってから,先生に備品購入の申請を行う.
\item 各個人で分担する作業には,今までの学習内容と,作業する上で新たに学習する必要がある領域の2つを含む.
\item KJ法を用いることで,多くの意見を引き出す.
\item ブレインストーミングを用いることで,効率的に関連性を見つける.
\end{itemize}
問題を解決するために,上記の具体例を活動課題とした.
\begin{flushright}
(※文責:宮嶋佑)
\end{flushright}
\section{到達レベル(目標)}
\subsection{ロボットの到達レベル(目標)}
グループ1では,店員の理想的な接客の「動き」を再現する.大きく2つの動作にわけ,さらにその中で2つの動き,合計で4つの動きの実装を目標として設定した.
\begin{description}
   \item[自発的な動作]\mbox{}
      \begin{itemize}
 \item  ロボットらしさを感じさせない,自然な動きの実装
 \item 押し付けがましくない,コンテンツの紹介を行う機能の実装\\
      \end{itemize}
   \item[客からの反応に答える動作]\mbox{}
      \begin{itemize}
      \item 客がいることを認識し,挨拶をする機能の実装
      \item ロボットが触られて,それに反応する機能の実装
         \end{itemize}
\end{description}
以上のロボット開発における目標を達成することで,客と自然なコミュニケーションを図ることのできるロボットの動きが実現できると考える.
\subsection{活動の到達レベル(目標)}
コロナウイルスの影響で,前年度とは全く違う活動方法となった.オンラインによる活動は新たな試みであり,わからない部分も多くある.また,対面の活動と比べ,コミュニケーションの取り方が非常に難しい.そのため,オンラインによる活動に目標を設定した.
\begin{itemize}
    \item 対面の活動よりも高頻度の報告と連絡をする.
    \item 音声だけではなく,画面共有やイラストを用い,コミュニケーションの相違をなくす.
    \item 活動を始める前にやるべきこと,終了する前に個人の進捗報告や意見交換の時間を設ける.
\end{itemize}
以上の活動における目標を常に意識することで,オンラインによる活動でも,円滑で間違いのない活動を行うことができると考える.
\begin{flushright}
(※文責:宮嶋佑)
\end{flushright}

\begin{comment}
メンバーに割り当てられた課題、課題を割り当てたプロセス
\end{comment}
\section{目標を達成するための割り当て}
意見交換を行い,以下の基準を提案しロボットを開発する上で,どの部分を担当するか,割り当てを行なった.
 \begin{itemize}
\item 各個人の得意分野
\item 各個人の興味のある分野
\item 各個人が習得したい技術
\item  作業負荷の均一性
\item パソコンのスペック(3D CAD を使用するには,ある程度のパソコンの動作環境が必要である.)
\end{itemize}
以上より,各個人の割り当ては以下のようになった.
\begin{description}
   \item[伊藤壱]\mbox{}
      \begin{itemize}
 \item  電子回路を中心に学習,設計
      \end{itemize}
   \item[木島拓海]\mbox{}
      \begin{itemize}
      \item リンク機構を中心とした,動きを実現する機構の学習,設計
         \end{itemize}
   \item[藤内悠]\mbox{}
      \begin{itemize}
      \item  歯車設計などを中心とした,動きを実現する機構の学習,設計
      \end{itemize}
   \item[宮嶋佑]\mbox{}   
   \begin{itemize}
   \item Fusion360による3D CADの学習,設計
   \end{itemize}
\end{description}
   また,割り当てごとに連携も行うことで,実現可能な動きか,実現するための変更点などの共有も行う.

\begin{flushright}
(※文責:宮嶋佑)
\end{flushright}

% ---第3章---
% 課題解決のプロセス
% solution_processing.tex
% 第3章「課題解決のプロセス」の内容 藤内
% ---Source---
\chapter{課題解決のプロセス}
\begin{comment}
各々のメンバーに割り当てられた課題解決の方法とプロセスを記述する。
\end{comment}
\section{プロジェクト内における課題の位置付け}
\subsection{解決するべき課題}
今プロジェクトにおいて主に以下の二つの課題の解決を中心とした。
\begin{itemize}
\item 店員ロボットにおける理想の「動き」
\item ハードウェアによる実現
\end{itemize}
\begin{flushright}
(※文責:藤内悠)
\end{flushright}
\subsection{課題の持つ背景}
\noindent
 現在の店員ロボットは決してそのスペックの低さ故に取り扱わない店舗が多いというわけでは無い。むしろ有り余る性能を持つにも関わらず普及しているとは言い難い。その理由として従来の店員ロボットの動きに無機質さがあり、そのためお客さんには近寄り難い雰囲気を与えたり、それを設置する店側としてはかえって不利益を被るということがあるのでは無いかと考察した。
グループ1においては概要でも述べた通り「動作」に着目し、理想的かつ簡易で表現できる動きとは何か、またそれを再現する上でハードウェアに必要な要素として機構や外観の作りを考察するに至った。
そのためにまずは店員が果たすべき振る舞いとはどのようなものかを実際の店員の観察や認知心理に基づく理由を含め考察し、具体的にどのような動作があるのかといったことを実現可能な範囲で挙げること、その動作を全て実現可能とする店員ロボットの外観及び内面機構の設計が最終的な課題となった。
\begin{flushright}
(※文責:藤内悠)
\end{flushright}
\section{課題解決の方法}
\subsection{理想の「動き」への考察とプロセス}
 まず理想の店員を考察するにあたり、一言に店員と言えどそのあり方は多種多様である。例えばお客さんの質問を聞いてそれに応えるものもあれば、お客さんとの対話を通じて抽象的な要望を現実的な答えとして提示するものもある。具体的にどのような場における店員をモデルにするべきかを話し合い定義をした。
理想的な店員という抽象的な概念を各々の経験談を用いて情報を共有し、それらの店員がなぜ理想的と感じたかを分析することとした。
また、理想的な振る舞いに対してどのような動作が所謂「無機質」と感じられてしまい敬遠されてしまうかについての考察と議論を重ねた。
そこで一つの原因として待機状態において全く動作しないことであった。人間の店員であれば、お客さんとのコミュニケーションがない状態であっても何かしらの動作がある。それは何かしらの作業に取り組んでというだけではなく、お客さんからのコミュニケーションを待機するような状態でもある。
現実における理想的な店員の振る舞いではお客さんがそのような何か作業をしている店員であっても助言や意見を求めて店員が受動的にコミュニケーションを始めることが多い。しかしロボット店員ではそれがなされないことが多い。
しかしながら、店員ロボットが何かしらの作業を行っていた場合に話しかけるお客さんはあまりいない。さらに言えば某店員ロボットは「僕とお話ししようよ」と音声を流しつつ待機しているにも関わらず奇異の目で見られたり興味はあっても近寄られないということが多い。
そこで直接的にコミュニケーションを促すのではなく抱いている興味からその店を訪れたお客さんがそのロボットを見て思わず何をしているのかと気になって近づくような待機状態の動作が解決策になると考察した。加えて待機状態だけではなく当然コミュニケーションを図っている際にも無機質さを感じさせないような細かな所作として2.3.1で挙げた4つの動作を元として設定することとした。
\begin{flushright}
(※文責:藤内悠)
\end{flushright}
\subsection{ハードウェアによる実現への考察とプロセス}
自然かつ無機質でないような動きを第2章でも触れた自発的な動作と客からの反応に答える動作、それぞれに二種類の動作の計4種類の動作の具体的な動作をフローとして明確にした。
一つ一つの動作においてどのような条件が必要か、またその条件を取得するためのセンサ等はどの程度必要かの目星をある程度付けGoogleJamboardを用いて図示をおこなった。
その際に動きを再現するための機構や制御を複数の案を出しつつ選定・改善を行い各動作の一連の処理を決定するに至った。またそれと並行しつつ動作を無理なく再現できるようにロボットのハードウェアの側面で可能な工夫や内部の機構等を図面として起こし、身近な素材による簡易版や動きの再現を確認することで解決に取り組むこととなった。
\begin{flushright}
(※文責:藤内悠)
\end{flushright}
% ---第4章---
% インターワーキング
\include{interworking}
% ---第5章---
% 結果
\chapter{結果}
\section{プロジェクトの成果}
本プロジェクトの前期の活動及び成果として、プロジェクト開始後,店員ロボットを制作するにあたっての問題点や役割についてディスカッションを行い「動き」「機能」「外見」の3つの観点に着目した.その中においてグループ1では「動き」に注目し,「動き」の動作を考察する上で,どんなに機能が優れていても,無機質な店員ロボットはいないのと同じであり,自分からお客さんに話しかけてもらうためのアプローチが必要であると考察した.また,今までのロボットらしくない動作を改善するため,実際の店員の動きを解析しました.そして,大きく自発動作と反応動作の2つ動作に分けることができると考察した.中間発表のアンケート結果より,「動き」の観点から課題解決を図る方向性に問題はなかった.質疑応答の時間に動画を流したことにより,肝心な質疑応答の時間が少なくなってしまい,質疑応答に回答できなかった人の疑問点を解消できず終わらせてしまった.他に,明確データなどがなく具体性に乏しさを感じさせるものとなってしまった.
\begin{flushright}
(※文責:木島拓海)
\end{flushright}
\section{プロジェクトにおける自分の役割}
本プロジェクトの前期の各個人の担当課題の成果は以下のようになった.
\begin{description}
   \item[伊藤壱]\mbox{}
   \item  電子回路を中心に学習,設計
      \begin{itemize}
     \item 直流回路と並列回路に纏わる電流、電圧の計算とコンダクタンスについて.
      \end{itemize}
      \begin{itemize}
     \item 容量とインダクタの特徴とそれを表す数式、回路記号について.
      \end{itemize}
      
   \item[木島拓海]\mbox{}
      \item リンク機構を中心とした,動きを実現する機構の学習,設計
      \begin{itemize}
      \item ロボット工作を購入し,組み立て機構や動きを学習
      \end{itemize}

   \item[藤内悠]\mbox{}
    \item  歯車設計などを中心とした,動きを実現する機構の学習,設計
      \begin{itemize}
      \item 腕の機構について
      \end{itemize}
      \begin{itemize}
      \item 遊星機構やユニバーサルジョイントを利用することの検討
      \end{itemize}
      \begin{itemize}
      \item be@brickの3dモデルを参考にしつつ内部の大まかな構想
      \end{itemize}

   \item[宮嶋佑]\mbox{}
     \item Fusion360による3D CADの学習,設計
      \begin{itemize}
      \item ロボットを3D CADで試作し,駆動域,モータが実装可能かの確認
      \item Fusion 360の学習:マスターガイドを参照
      \end{itemize}
\end{description}

\begin{flushright}
(※文責:木島拓海)
\end{flushright}

\section{今後の課題}
本プロジェクトのグループ1の目的である「先手を打つコミュニケーション」を実現するために,待ち動作を充実させることでもう一つのコンセプトである「先手を打つコミュニケーション」と対立することにならないように,待ち動作の工夫が必要である.さらに,ロボットの頭を撫でるという行為はコロナウイルス の流行している中では受け入れられづらい可能性があり,工夫や改善する必要がある.また,店員が果たすべき振る舞いとはどのようなものかを実際の店員の観察や認知心理に基づく理由を含め考察し,具体的にどのような動作があるのかといったことを実現可能な範囲で挙げること,その動作を全て実現可能とする店員ロボットの外観及び内面機構の設計を行う.今後の展望としては,3Dモデル・電子回路図・機械設計図を作成する.また,発砲スチロールを用いてプロトタイプの製作にあたる.その後,必要部品を調達・印刷し実際にロボットを組み立て,完成後、見直し作業・再設計・チューニングを行なっていく.
\begin{flushright}
(※文責:木島拓海)
\end{flushright}
% 第6章
% まとめ
% reference.tex
% 参考文献
%図書の場合
%著者 図書名.出版人,出版社,出版年
% ---Source--
\begin{thebibliography}{99}

\bibitem{い}伊藤茂
メカニズムの事典,村上和夫,株式会社オーム社,2016

\bibitem{え} 遠藤敏夫 わかる電子工作の基本100

\bibitem{お}小原照記,藤村祐爾 Fusion360マスターガイドベーシック編.柳沢淳一,久保田賢二,株式会社ソーテック社,2018

\bibitem{ば} 馬場政勝 ロボットキットで学ぶ機械工学.星正明,株式会社工学社,2018

\end{thebibliography}
%参考文献
% appendix.tex
% 付録
% ---Source---
\appendix
\chapter{}
\section{課題解決のための技術(新規取得)}
      \begin{itemize}
 \item  Fusion360をもちいた,3D CAD
 \item Arduinoによる各種センサの活用方法
      \end{itemize}
\section{課題解決のための技術(講義)}
      \begin{itemize}
 \item Arduinoによるモータの駆動(情報表現入門)
 \item KJ法(Communication I\hspace{-.1em}I\hspace{-.1em}I)
 \item ブレインストーミング(Communication I\hspace{-.1em}I\hspace{-.1em}I)
 \item リンク機構(ロボットの科学技術)
  \item  ロボット用センサ(ロボットの科学技術)

 \item ブレインストーミング
      \end{itemize}
\section{相互評価}
\begin{description}
   \item[伊藤壱]\mbox{}
      \begin{itemize}
       \item コメンター氏名:宮嶋佑\\
        \item コメンター氏名:藤内悠\\プロジェクトのリーダーを平行しつつグループの作業方針についても中心的な役割を果たし、方向性を指し示すことが多かったと思います。group1に限らずプロジェクト全体が計画性をもって作業できたのは伊藤君のおかげです。
 \item  コメンター氏名:宮嶋佑\\プロジェクトのリーダーを務めていながらも,グループ内でも率先してアイデアを出したり,意見を出していました.また,任された学習領域の電子回路の部分では,積極的に学習を進めて行ったり,知識の共有を行なっていました.\\
      \end{itemize}
         \item[木島拓海]\mbox{}
      \begin{itemize}
             \item コメンター氏名:伊藤壱\\
             木島君はどんな状況でも軽快に話をしてくれるので,多くの班員がその雰囲気に和まされたと思います.これからも持ち前の気前の良さでプロジェクトを支えてほしいと思います.
        \item コメンター氏名:藤内悠\\木島君は話し合いの場で方向性の確認や脱線をしないように適宜指摘をしてくれたと思います。また、活動の際に多角的な指摘で意見を出してくれた為、様々な間違いを早期に発見し非常に助かる場面が多くありました。
 \item  コメンター氏名:宮嶋佑\\グループ内での中間発表のスライド資料作りでは,的確な意見がもらえて助かりました.また,必要となった学習領域の割り当ての際,率先してそその学習領域に就いていました.\\
      \end{itemize}
         \item[藤内悠]\mbox{}
      \begin{itemize}
 \item  コメンター氏名:宮嶋佑\\ロボットの動きを考える時に,積極的に図示して説明していて,納得させられるところが多かったです.また,意見交換をする際に,率先して意見交換の場(docsなど)を開いてくれるので,円滑に物事を進めることができました.\\
  \item  コメンター氏名:伊藤壱\\
  藤内君は班員として励むだけではなく,プロジェクト全体の視点を持って熱心に取り組んでいました.その姿勢をとても尊敬しています.私がプロジェクトを進める上でとても助けられることが多かったと感謝しています.
      \end{itemize}
         \item[宮嶋佑]\mbox{}
      \begin{itemize}
      \item コメンター氏名:伊藤壱\\
   とても頑張っていたと思います.宮島さんの論理的な意見に何度も助けられました.責任感が強く最後まで仕事をやり抜く力を見習いたいと思います.
        \item コメンター氏名:宮嶋佑\\
        \item コメンター氏名:藤内悠\\話し合いや全体での作業が滞ってしまいそうな時に革新的なアイディアを提示し、参考になりそうな情報や資料を前もって準備する姿勢にはグループ全体として助けられたことが多くありました。
      \end{itemize}
\end{description}
% 付録
\end{document}